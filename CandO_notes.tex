\documentclass[12pt]{article}
\usepackage{amssymb}
\usepackage{amsfonts}
\usepackage{amsmath} 
\usepackage{bm}
\usepackage{graphicx}
\usepackage{cancel}
\usepackage{enumitem}
\usepackage{fancyhdr}
\usepackage{fancyvrb}
\usepackage[paperwidth=8.5in,paperheight=11in,margin=1in]{geometry}

\pagestyle{fancy}
\lhead{Peter Williams}
\chead{Carroll and Ostlie Notes}
\rhead{\today}

\begin{document}
\section*{Chapter 1}

\section*{Chapter 2}

\section*{Chapter 3}
\begin{itemize}
\item Stellar parallax: 
\begin{equation}
d = \frac{1\text{ AU}}{\tan p} \approx \frac{1}{p}\text{ AU}
\end{equation}
\begin{itemize}
\item parsec: distance when parallax angle is 1 arcsecond.
\item Hipparcos (ESA): accuracies approaching 0.001$^{\prime\prime}$
\item Gaia: accuracies approaching 4 microarcsec
\end{itemize}
\item apparent magnitude:
\begin{itemize}
\item difference of 5 magnitudes = factor of 100 difference in brightness $\Longrightarrow$ $100^{1/5} = 2.512$
\item Solar constant: 1.4 kW m$^{-2}$
\item
\begin{equation}
\frac{F_2}{F_1} = 100^{(m_1-m_2)/5} \Longrightarrow m_1 - m_2 = -2.5\log_{10}\left(\frac{F_1}{F_2}\right)
\end{equation}
\end{itemize}
\item absolute magnitude:
\begin{itemize}
\item apparent magnitude at 10 kpc
\item distance modulus
\begin{align*}
100^{(m - M)/5} = \frac{F_{10}}{F} = \left(\frac{d}{10\text{ pc}}\right)^2 &\Longrightarrow d = 10^{(m - M + 5)/5}\text{ pc} \\
&\Longrightarrow m - M = 5\log_{10}(d) - 5 = 5\log_{10}\left(\frac{d}{10\text{ pc}}\right)
\end{align*}
\end{itemize}
\item Double slit experiment:
\begin{itemize}
\item spacing $d$ between slits, distance $L$ to screen. At angle $\theta$, extra path length of $d\sin\theta$ for far slit $\Longrightarrow$ bright if $d\sin\theta = n\lambda$, dark if $d\sin\theta = (n - \frac{1}{2})\lambda$.
\end{itemize}
\item Poynting vector:
\begin{equation}
\bm{S} = \frac{1}{\mu_0}\bm{E}\times\bm{B}\text{ (SI)} = \frac{c}{4\pi}\bm{E}\times\bm{B}\text{ (cgs)}
\end{equation}
\begin{itemize}
\item radiation pressure:
\begin{align*}
F_{rad} &= \frac{\langle S\rangle A}{c}\cos\theta~~~~~~~~~\text{(absorption)}\\
F_{rad} &= \frac{2\langle S\rangle A}{c}\cos^2\theta~~~~~~\text{(reflection)}
\end{align*}
\end{itemize}
\item Blackbodies:
\begin{itemize}
\item Wien's displacement law:
\begin{equation}
\lambda_\text{max} T = 0.002897755\text{ m K}
\end{equation}
\item Stefan-Boltzmann equation:
\begin{equation}
L = A\sigma T^4,~~~~\sigma = 5.67\times 10^{-8}\text{ W m}^{-2}\text{K}^{-4}.
\end{equation}
\item Rayleigh-Jeans law: works for $\lambda$ large. Take oven with distance $L$ between walls, then can have standing waves with wavelength $\lambda = 2L/1, 2L/2, 2L/3, 2L/4, 2L/5, \dots,$ each wavelength with energy $kT$. Then (DERIVATION --- LEARN)
\begin{equation}
B_\lambda(T)\approx\frac{2ckT}{\lambda^4}
\end{equation}
\item Wien's approximation: works for $\lambda$ small. (DERIVATION --- LEARN)
\begin{equation}
B_\lambda(T) \approx a \lambda^{-5} e^{-b/\lambda T}
\end{equation}
\item Planck:
\begin{equation}
B_\lambda(T) = \frac{a/\lambda^5}{e^{b/kT} - 1}
\end{equation}
Assume energy must be quantized: $nh\nu$. (DERIVATION --- LEARN)
\begin{equation}
B_\lambda(T) = \frac{2hc^2/\lambda^5}{e^{hc/\lambda k T} - 1}
\end{equation}
$\lambda = c/\nu$, so $d\lambda = c/\nu^2 d\nu$:
\begin{equation}
B_\nu(T) = \frac{2 h \nu^3/c^2}{e^{h\nu/k T} - 1}
\end{equation}
\end{itemize}
\item Color index
\begin{align*}
U - B &= M_U - M_B \\
B - V &= M_B - M_V
\end{align*}
\begin{itemize}
\item Bolometric correction:
\begin{equation}
BC = m_\text{bol} - V = M_\text{bol} - M_V
\end{equation}
\item Sensitivity function $S(\lambda)$
\begin{equation}
U = -2.5\log_{10}\left(\int_0^\infty F_\lambda S_U d\lambda\right)+C_U
\end{equation}
$C_U$ chosen such that Vega has magnitude zero in each filter.
\begin{equation}
U - B = -2.5\log_{10}\left(\frac{\int F_\lambda S_U d\lambda}{\int F_\lambda S_B d\lambda}\right)+C_{U-B},~~C_{U-B} \equiv C_U - C_B
\end{equation}
\item *** Color index does not depend on distance, so it is a measure solely of the temperature of a model blackbody star
\end{itemize}
\end{itemize}

\section*{Chapter 4}
\begin{itemize}
\item Galilean transformations:
\begin{eqnarray}
x^\prime &= &x - ut \\
y^\prime &= &y \\
z^\prime &= &z \\
t^\prime &= &t
\end{eqnarray}
\begin{equation}
\bf{v}^\prime = \bf{v} - \bf{u},~~\bf{a}^\prime = \bf{a}
\end{equation}

\item Michelson-Morley Experiment - no ether

\item Lorentz transformations:
\begin{eqnarray}
x^\prime &= &\frac{x - ut}{\sqrt{1 - u^2/c^2}} \\
y^\prime &= &y \\
z^\prime &= &z \\
t^\prime &= &\frac{t - ux/c^2}{\sqrt{1 - u^2/c^2}}
\end{eqnarray}
\begin{equation}
\gamma \equiv \frac{1}{\sqrt{1-u^2/c^2}}
\end{equation}
\begin{itemize}
\item Time dilation:
\begin{equation}
\Delta t_\text{moving} = \frac{\Delta t_\text{rest}}{\sqrt{1-u^2/c^2}}
\end{equation}
\item Length contraction:
\begin{equation}
L_\text{moving} = L_\text{rest}\sqrt{1-u^2/c^2}
\end{equation}
\item Relativistic Doppler shift:
\begin{equation}
\nu_\text{obs} = \frac{\nu_\text{rest}\sqrt{1-u^2/c^2}}{1+(u/c)\cos\theta} = \frac{\nu_\text{rest}\sqrt{1-u^2/c^2}}{1+v_r/c}
\end{equation}
\begin{equation}
\Longrightarrow \nu_\text{obs} = \nu_\text{rest}\sqrt{\frac{1 - v_r/c}{1+v_r/c}} ~~~~~~~(\text{radial motion})
\end{equation}
***Note that there is still a transverse Doppler shift
\end{itemize}
\item Cosmological redshift:
\begin{equation}
z\equiv \frac{\lambda_\text{obs} - \lambda_\text{rest}}{\lambda_\text{rest}} = \frac{\Delta \lambda}{\lambda_\text{rest}}
\end{equation}
\begin{itemize}
\item For $u/c << 1$, $z\approx v_r/c$.
\end{itemize}
\item Velocity transformations:
\begin{eqnarray}
v_x^\prime &= &\frac{v_x - u}{1 - uv_x/c^2} \\
v_y^\prime &= &\frac{v_y\sqrt{1 - u^2/c^2}}{1 - uv_x/c^2} \\
v_z^\prime &= &\frac{v_z\sqrt{1 - u^2/c^2}}{1 - uv_x/c^2}
\end{eqnarray}
\item Relativistic beaming: Consider a light source moving in the positive $x$-direction with relativistic speed $u$. Consider light ray traveling in the $y$ direction (at speed $c$) in the light source's rest frame. Then
\begin{equation}
v_x = \frac{v_x^\prime + u}{1 - uv_x^\prime/c^2} = u
\end{equation}
\begin{equation}
v_y = \frac{v_y^\prime\sqrt{1-u^2/c^2}}{1+uv_x^\prime/c^2} = c\sqrt{1 - u^2/c^2}
\end{equation}
\begin{equation}
v_z = \frac{v_z^\prime\sqrt{1 - u^2/c^2}}{1 + u v_x^\prime/c^2} = 0
\end{equation}
\begin{equation}
\Longrightarrow \sin\theta = v_y/c = \gamma^{-1}
\end{equation}

\item Relativistic momentum:
\begin{equation}
{\bf p} = \frac{m{\bf v}}{\sqrt{1 - v^2/c^2}} = \gamma m {\bf v}
\end{equation}

\item Relativstic kinetic energy:
\begin{equation}
K = \int_{x_i}^{x_f} F dx = \int_{x_i}^{x_f}\frac{dp}{dt}dx =  \int_{p_i}^{p_f}\frac{dx}{dt}{dp} = \int_{p_i}^{p_f} v dp
\end{equation}
Using $p_i = 0$ and integrating by parts, 
\begin{align}
K &= p_fv_f - \int_0^{v_f} p dv \\
&= \frac{mv_f^2}{\sqrt{1 - v_f^2/c^2}} - \int_0^{v_f}\frac{m v}{\sqrt{1 - v^2/c^2}}dv \\
&= \frac{mv_f^2}{\sqrt{1 - v_f^2/c^2}} + mc^2\left(\sqrt{1- v_f^2/c^2} - 1\right) \\
&= mc^2\left(\frac{v_f^2/c^2 + 1 - v_f^2/c^2 - \sqrt{1 - v_f^2/c^2}}{\sqrt{1 - v_f^2/c^2}}\right) \\
&= mc^2\left(\frac{1}{\sqrt{1 - v_f^2/c^2}} - 1\right) = mc^2(\gamma - 1)
\end{align}
\item Total relativistic energy:
\begin{equation}
E = \frac{ mc^2}{\sqrt{1 - v^2/c^2}} = \gamma m c^2
\end{equation}
\begin{equation}
E^2 = p^2 c^2 + m^2 c^4
\end{equation}
\item Example: Two particles of mass $m$ and speed $v$ collide and merge to particle of mass $M$:\\
Total energy of system is initially
\begin{equation}
E_{\text{sys},i} = 2\gamma m c^2
\end{equation}
and momentum is ${\bf p}_\text{sys}  =0$. After collision, $E_{\text{sys},f} = M c^2$. Equating initial and final energies:
\begin{equation}
2\gamma m c^2 = M c^2 \Longrightarrow M = 2 \gamma m,
\end{equation}
so \begin{equation}
\Delta m = 2m(\gamma - 1).
\end{equation}
This is equal to the kinetic energy lost/$c^2$:
\begin{equation}
\frac{2 m c^2(\gamma - 1) - 0}{c^2} = \Delta m
\end{equation}

\end{itemize}

\section*{Chapter 5}

\section*{Chapter 6}

\section*{Chapter 7}

\section*{Chapter 8}

\section*{Chapter 9}

\section*{Chapter 10}

\end{document}