\documentclass[12pt]{article}
\usepackage{amssymb}
\usepackage{amsfonts}
\usepackage{amsmath} 
\usepackage{bm}
\usepackage{graphicx}
\usepackage{cancel}
\usepackage{enumitem}
\usepackage{fancyhdr}
\usepackage{fancyvrb}
\usepackage[paperwidth=8.5in,paperheight=11in,margin=1in]{geometry}

\pagestyle{fancy}
\lhead{Peter Williams}
\chead{Reverberation Mapping Project Notes}
\rhead{\today}

\begin{document}
\section*{Chapter 1}

\section*{Chapter 2}

\section*{Chapter 3}
\begin{itemize}
\item Stellar parallax: 
\begin{equation}
d = \frac{1\text{ AU}}{\tan p} \approx \frac{1}{p}\text{ AU}
\end{equation}
\begin{itemize}
\item parsec: distance when parallax angle is 1 arcsecond.
\item Hipparcos (ESA): accuracies approaching 0.001$^{\prime\prime}$
\item Gaia: accuracies approaching 4 microarcsec
\end{itemize}
\item apparent magnitude:
\begin{itemize}
\item difference of 5 magnitudes = factor of 100 difference in brightness $\Longrightarrow$ $100^{1/5} = 2.512$
\item Solar constant: 1.4 kW m$^{-2}$
\item
\begin{equation}
\frac{F_2}{F_1} = 100^{(m_1-m_2)/5} \Longrightarrow m_1 - m_2 = -2.5\log_{10}\left(\frac{F_1}{F_2}\right)
\end{equation}
\end{itemize}
\item absolute magnitude:
\begin{itemize}
\item apparent magnitude at 10 kpc
\item distance modulus
\begin{align*}
100^{(m - M)/5} = \frac{F_{10}}{F} = \left(\frac{d}{10\text{ pc}}\right)^2 &\Longrightarrow d = 10^{(m - M + 5)/5}\text{ pc} \\
&\Longrightarrow m - M = 5\log_{10}(d) - 5 = 5\log_{10}\left(\frac{d}{10\text{ pc}}\right)
\end{align*}
\end{itemize}
\item Double slit experiment:
\begin{itemize}
\item spacing $d$ between slits, distance $L$ to screen. At angle $\theta$, extra path length of $d\sin\theta$ for far slit $\Longrightarrow$ bright if $d\sin\theta = n\lambda$, dark if $d\sin\theta = (n - \frac{1}{2})\lambda$.
\end{itemize}
\item Poynting vector:
\begin{equation}
\bm{S} = \frac{1}{\mu_0}\bm{E}\times\bm{B}\text{ (SI)} = \frac{c}{4\pi}\bm{E}\times\bm{B}\text{ (cgs)}
\end{equation}
\begin{itemize}
\item radiation pressure:
\begin{align*}
F_{rad} &= \frac{\langle S\rangle A}{c}\cos\theta\,\,\,\,\,\,\,\,\,\,\text{(absorption)}\\
F_{rad} &= \frac{2\langle S\rangle A}{c}\cos^2\theta\,\,\,\,\,\,\,\,\,\,\text{(reflection)}
\end{align*}
\end{itemize}
\item Blackbodies:
\begin{itemize}
\item Wien's displacement law:
\begin{equation}
\lambda_\text{max} T = 0.002897755\text{ m K}
\end{equation}
\item Stefan-Boltzmann equation:
\begin{equation}
L = A\sigma T^4,\,\,\,\,\,\,\sigma = 5.67\times 10^{-8}\text{ W m}^{-2}\text{K}^{-4}.
\end{equation}
\item Rayleigh-Jeans law: works for $\lambda$ large. Take oven with distance $L$ between walls, then can have standing waves with wavelength $\lambda = 2L/1, 2L/2, 2L/3, 2L/4, 2L/5, \dots,$ each wavelength with energy $kT$. Then (DERIVATION --- LEARN)
\begin{equation}
B_\lambda(T)\approx\frac{2ckT}{\lambda^4}
\end{equation}
\item Wien's approximation: works for $\lambda$ small. (DERIVATION --- LEARN)
\begin{equation}
B_\lambda(T) \approx a \lambda^{-5} e^{-b/\lambda T}
\end{equation}
\item Planck:
\begin{equation}
B_\lambda(T) = \frac{a/\lambda^5}{e^{b/kT} - 1}
\end{equation}
Assume energy must be quantized: $nh\nu$. (DERIVATION --- LEARN)
\begin{equation}
B_\lambda(T) = \frac{2hc^2/\lambda^5}{e^{hc/\lambda k T} - 1}
\end{equation}
$\lambda = c/\nu$, so $d\lambda = c/\nu^2 d\nu$:
\begin{equation}
B_\nu(T) = \frac{2 h \nu^3/c^2}{e^{h\nu/k T} - 1}
\end{equation}
\end{itemize}
\item Color index
\begin{align*}
U - B &= M_U - M_B \\
B - V &= M_B - M_V
\end{align*}
\begin{itemize}
\item Bolometric correction:
\begin{equation}
BC = m_\text{bol} - V = M_\text{bol} - M_V
\end{equation}
\item Sensitivity function $S(\lambda)$
\begin{equation}
U = -2.5\log_{10}\left(\int_0^\infty F_\lambda S_U d\lambda\right)+C_U
\end{equation}
$C_U$ chosen such that Vega has magnitude zero in each filter.
\begin{equation}
U - B = -2.5\log_{10}\left(\frac{\int F_\lambda S_U d\lambda}{\int F_\lambda S_B d\lambda}\right)+C_{U-B},\,\,\,\,\,\,C_{U-B} \equiv C_U - C_B
\end{equation}
\item *** Color index does not depend on distance, so it is a measure solely of the temperature of a model blackbody star
\end{itemize}
\end{itemize}

\section*{Chapter 4}

\section*{Chapter 5}

\section*{Chapter 6}

\section*{Chapter 7}

\section*{Chapter 8}

\section*{Chapter 9}

\section*{Chapter 10}

\end{document}